% Options for packages loaded elsewhere
\PassOptionsToPackage{unicode}{hyperref}
\PassOptionsToPackage{hyphens}{url}
%
\documentclass[
]{article}
\usepackage{amsmath,amssymb}
\usepackage{iftex}
\ifPDFTeX
  \usepackage[T1]{fontenc}
  \usepackage[utf8]{inputenc}
  \usepackage{textcomp} % provide euro and other symbols
\else % if luatex or xetex
  \usepackage{unicode-math} % this also loads fontspec
  \defaultfontfeatures{Scale=MatchLowercase}
  \defaultfontfeatures[\rmfamily]{Ligatures=TeX,Scale=1}
\fi
\usepackage{lmodern}
\ifPDFTeX\else
  % xetex/luatex font selection
\fi
% Use upquote if available, for straight quotes in verbatim environments
\IfFileExists{upquote.sty}{\usepackage{upquote}}{}
\IfFileExists{microtype.sty}{% use microtype if available
  \usepackage[]{microtype}
  \UseMicrotypeSet[protrusion]{basicmath} % disable protrusion for tt fonts
}{}
\makeatletter
\@ifundefined{KOMAClassName}{% if non-KOMA class
  \IfFileExists{parskip.sty}{%
    \usepackage{parskip}
  }{% else
    \setlength{\parindent}{0pt}
    \setlength{\parskip}{6pt plus 2pt minus 1pt}}
}{% if KOMA class
  \KOMAoptions{parskip=half}}
\makeatother
\usepackage{xcolor}
\usepackage[margin=1in]{geometry}
\usepackage{color}
\usepackage{fancyvrb}
\newcommand{\VerbBar}{|}
\newcommand{\VERB}{\Verb[commandchars=\\\{\}]}
\DefineVerbatimEnvironment{Highlighting}{Verbatim}{commandchars=\\\{\}}
% Add ',fontsize=\small' for more characters per line
\usepackage{framed}
\definecolor{shadecolor}{RGB}{248,248,248}
\newenvironment{Shaded}{\begin{snugshade}}{\end{snugshade}}
\newcommand{\AlertTok}[1]{\textcolor[rgb]{0.94,0.16,0.16}{#1}}
\newcommand{\AnnotationTok}[1]{\textcolor[rgb]{0.56,0.35,0.01}{\textbf{\textit{#1}}}}
\newcommand{\AttributeTok}[1]{\textcolor[rgb]{0.13,0.29,0.53}{#1}}
\newcommand{\BaseNTok}[1]{\textcolor[rgb]{0.00,0.00,0.81}{#1}}
\newcommand{\BuiltInTok}[1]{#1}
\newcommand{\CharTok}[1]{\textcolor[rgb]{0.31,0.60,0.02}{#1}}
\newcommand{\CommentTok}[1]{\textcolor[rgb]{0.56,0.35,0.01}{\textit{#1}}}
\newcommand{\CommentVarTok}[1]{\textcolor[rgb]{0.56,0.35,0.01}{\textbf{\textit{#1}}}}
\newcommand{\ConstantTok}[1]{\textcolor[rgb]{0.56,0.35,0.01}{#1}}
\newcommand{\ControlFlowTok}[1]{\textcolor[rgb]{0.13,0.29,0.53}{\textbf{#1}}}
\newcommand{\DataTypeTok}[1]{\textcolor[rgb]{0.13,0.29,0.53}{#1}}
\newcommand{\DecValTok}[1]{\textcolor[rgb]{0.00,0.00,0.81}{#1}}
\newcommand{\DocumentationTok}[1]{\textcolor[rgb]{0.56,0.35,0.01}{\textbf{\textit{#1}}}}
\newcommand{\ErrorTok}[1]{\textcolor[rgb]{0.64,0.00,0.00}{\textbf{#1}}}
\newcommand{\ExtensionTok}[1]{#1}
\newcommand{\FloatTok}[1]{\textcolor[rgb]{0.00,0.00,0.81}{#1}}
\newcommand{\FunctionTok}[1]{\textcolor[rgb]{0.13,0.29,0.53}{\textbf{#1}}}
\newcommand{\ImportTok}[1]{#1}
\newcommand{\InformationTok}[1]{\textcolor[rgb]{0.56,0.35,0.01}{\textbf{\textit{#1}}}}
\newcommand{\KeywordTok}[1]{\textcolor[rgb]{0.13,0.29,0.53}{\textbf{#1}}}
\newcommand{\NormalTok}[1]{#1}
\newcommand{\OperatorTok}[1]{\textcolor[rgb]{0.81,0.36,0.00}{\textbf{#1}}}
\newcommand{\OtherTok}[1]{\textcolor[rgb]{0.56,0.35,0.01}{#1}}
\newcommand{\PreprocessorTok}[1]{\textcolor[rgb]{0.56,0.35,0.01}{\textit{#1}}}
\newcommand{\RegionMarkerTok}[1]{#1}
\newcommand{\SpecialCharTok}[1]{\textcolor[rgb]{0.81,0.36,0.00}{\textbf{#1}}}
\newcommand{\SpecialStringTok}[1]{\textcolor[rgb]{0.31,0.60,0.02}{#1}}
\newcommand{\StringTok}[1]{\textcolor[rgb]{0.31,0.60,0.02}{#1}}
\newcommand{\VariableTok}[1]{\textcolor[rgb]{0.00,0.00,0.00}{#1}}
\newcommand{\VerbatimStringTok}[1]{\textcolor[rgb]{0.31,0.60,0.02}{#1}}
\newcommand{\WarningTok}[1]{\textcolor[rgb]{0.56,0.35,0.01}{\textbf{\textit{#1}}}}
\usepackage{graphicx}
\makeatletter
\def\maxwidth{\ifdim\Gin@nat@width>\linewidth\linewidth\else\Gin@nat@width\fi}
\def\maxheight{\ifdim\Gin@nat@height>\textheight\textheight\else\Gin@nat@height\fi}
\makeatother
% Scale images if necessary, so that they will not overflow the page
% margins by default, and it is still possible to overwrite the defaults
% using explicit options in \includegraphics[width, height, ...]{}
\setkeys{Gin}{width=\maxwidth,height=\maxheight,keepaspectratio}
% Set default figure placement to htbp
\makeatletter
\def\fps@figure{htbp}
\makeatother
\setlength{\emergencystretch}{3em} % prevent overfull lines
\providecommand{\tightlist}{%
  \setlength{\itemsep}{0pt}\setlength{\parskip}{0pt}}
\setcounter{secnumdepth}{-\maxdimen} % remove section numbering
\ifLuaTeX
  \usepackage{selnolig}  % disable illegal ligatures
\fi
\IfFileExists{bookmark.sty}{\usepackage{bookmark}}{\usepackage{hyperref}}
\IfFileExists{xurl.sty}{\usepackage{xurl}}{} % add URL line breaks if available
\urlstyle{same}
\hypersetup{
  pdftitle={Lesson3\_Sandbox},
  pdfauthor={Jay Aher},
  hidelinks,
  pdfcreator={LaTeX via pandoc}}

\title{Lesson3\_Sandbox}
\author{Jay Aher}
\date{2023-06-21}

\begin{document}
\maketitle

\hypertarget{background-for-this-activity}{%
\subsection{Background for this
activity}\label{background-for-this-activity}}

Welcome to the sandbox! This activity is going to provide you with the
opportunity to preview some of the cool things you can do in \texttt{R}
that you will be learning in this course. You will learn more about
working with packages and data and try out some important functions.

In this activity, you are going to install and load \texttt{R} packages;
practice using functions to view, clean, and visualize data; and learn
more about using \texttt{R\ markdown} to document your analysis.
\texttt{R} is a powerful tool that can do a lot of different things;
this sandbox activity will help you get more comfortable using
\texttt{R} while demonstrating some of its functions in action. In later
activities, you will also get the opportunity to write your own R code!

\hypertarget{step-1-using-r-packages}{%
\subsection{\texorpdfstring{Step 1: Using
\texttt{R\ packages}}{Step 1: Using R packages}}\label{step-1-using-r-packages}}

\texttt{Packages} are a key part of working with \texttt{R.}They contain
bundles of code called \texttt{functions} that allow you to perform a
wide range of tasks in \texttt{R.} Some of them even contain datasets
that you can use to practice the skills you have been learning
throughout this course.

Some \texttt{packages} are installed by default, but many others can be
downloaded from an external source such as the Comprehensive R Archive
Network, or CRAN.

In this activity, you will be using a package called \texttt{tidyverse.}
The \texttt{tidyverse} package is actually a collection individual
\texttt{packages} that can help you perform a wide variety of analysis
tasks.

To install the \texttt{tidyverse} package, execute the code in the code
chunk below by clicking on the green arrow button in the top right
corner. When you execute a code chunk in RMarkdown, the output will
appear in the .rmd area and your console.

\begin{Shaded}
\begin{Highlighting}[]
\FunctionTok{install.packages}\NormalTok{(}\StringTok{"tidyverse"}\NormalTok{)}
\end{Highlighting}
\end{Shaded}

\begin{verbatim}
## Installing tidyverse [2.0.0] ...
##  OK [linked cache in 2.4 milliseconds]
## * Installed 1 package in 3.1 seconds.
\end{verbatim}

Once a package is installed, you can load it by running the
\texttt{library()} function with the package name inside the
parentheses, like this:

\begin{Shaded}
\begin{Highlighting}[]
\FunctionTok{library}\NormalTok{(tidyverse)}
\end{Highlighting}
\end{Shaded}

\begin{verbatim}
## -- Attaching core tidyverse packages ------------------------ tidyverse 2.0.0 --
## v dplyr     1.1.2     v readr     2.1.4
## v forcats   1.0.0     v stringr   1.5.0
## v ggplot2   3.4.2     v tibble    3.2.1
## v lubridate 1.9.2     v tidyr     1.3.0
## v purrr     1.0.1     
## -- Conflicts ------------------------------------------ tidyverse_conflicts() --
## x dplyr::filter() masks stats::filter()
## x dplyr::lag()    masks stats::lag()
## i Use the conflicted package (<http://conflicted.r-lib.org/>) to force all conflicts to become errors
\end{verbatim}

Installing and loading the \texttt{tidyverse} package may take a few
minutes-- be sure to wait for it to finish running before moving on to
the next steps!

Once the chunk above has finished running, you will get a report that
summarizes what packages were loaded because you ran the
\texttt{library()} function. The report will also let you know you if
there are any \texttt{functions} that have a conflict, but you don't
need to worry about that for now.

Now that you have loaded an \texttt{R\ package,} you can start exploring
some data.

\hypertarget{step-2-viewing-data}{%
\section{Step 2: Viewing data}\label{step-2-viewing-data}}

Many of the \texttt{tidyverse} packages contain sample datasets that you
can use to practice your \texttt{R} skills. The \texttt{diamonds}
dataset in the \texttt{ggplot2} package is a great example for
previewing \texttt{R} functions.

Because you already loaded this package in the last step, the
\texttt{diamonds} dataset is ready for you to use.

One common function you can use to preview the data is the
\texttt{head()} function, which displays the columns and the first
several rows of data. You can test out how the \texttt{head()} function
works by running the chunk below:

\begin{Shaded}
\begin{Highlighting}[]
\FunctionTok{head}\NormalTok{(diamonds)}
\end{Highlighting}
\end{Shaded}

\begin{verbatim}
## # A tibble: 6 x 10
##   carat cut       color clarity depth table price     x     y     z
##   <dbl> <ord>     <ord> <ord>   <dbl> <dbl> <int> <dbl> <dbl> <dbl>
## 1  0.23 Ideal     E     SI2      61.5    55   326  3.95  3.98  2.43
## 2  0.21 Premium   E     SI1      59.8    61   326  3.89  3.84  2.31
## 3  0.23 Good      E     VS1      56.9    65   327  4.05  4.07  2.31
## 4  0.29 Premium   I     VS2      62.4    58   334  4.2   4.23  2.63
## 5  0.31 Good      J     SI2      63.3    58   335  4.34  4.35  2.75
## 6  0.24 Very Good J     VVS2     62.8    57   336  3.94  3.96  2.48
\end{verbatim}

In addition to \texttt{head()} there are a number of other useful
functions you can use to summarize or preview the data. For example, the
\texttt{str()} and \texttt{glimpse()} functions will both return
summaries of each column in your data arranged horizontally. You can try
out these two functions by running the code chunks below:

\begin{Shaded}
\begin{Highlighting}[]
\FunctionTok{str}\NormalTok{(diamonds)}
\end{Highlighting}
\end{Shaded}

\begin{verbatim}
## tibble [53,940 x 10] (S3: tbl_df/tbl/data.frame)
##  $ carat  : num [1:53940] 0.23 0.21 0.23 0.29 0.31 0.24 0.24 0.26 0.22 0.23 ...
##  $ cut    : Ord.factor w/ 5 levels "Fair"<"Good"<..: 5 4 2 4 2 3 3 3 1 3 ...
##  $ color  : Ord.factor w/ 7 levels "D"<"E"<"F"<"G"<..: 2 2 2 6 7 7 6 5 2 5 ...
##  $ clarity: Ord.factor w/ 8 levels "I1"<"SI2"<"SI1"<..: 2 3 5 4 2 6 7 3 4 5 ...
##  $ depth  : num [1:53940] 61.5 59.8 56.9 62.4 63.3 62.8 62.3 61.9 65.1 59.4 ...
##  $ table  : num [1:53940] 55 61 65 58 58 57 57 55 61 61 ...
##  $ price  : int [1:53940] 326 326 327 334 335 336 336 337 337 338 ...
##  $ x      : num [1:53940] 3.95 3.89 4.05 4.2 4.34 3.94 3.95 4.07 3.87 4 ...
##  $ y      : num [1:53940] 3.98 3.84 4.07 4.23 4.35 3.96 3.98 4.11 3.78 4.05 ...
##  $ z      : num [1:53940] 2.43 2.31 2.31 2.63 2.75 2.48 2.47 2.53 2.49 2.39 ...
\end{verbatim}

\begin{Shaded}
\begin{Highlighting}[]
\FunctionTok{glimpse}\NormalTok{(diamonds)}
\end{Highlighting}
\end{Shaded}

\begin{verbatim}
## Rows: 53,940
## Columns: 10
## $ carat   <dbl> 0.23, 0.21, 0.23, 0.29, 0.31, 0.24, 0.24, 0.26, 0.22, 0.23, 0.~
## $ cut     <ord> Ideal, Premium, Good, Premium, Good, Very Good, Very Good, Ver~
## $ color   <ord> E, E, E, I, J, J, I, H, E, H, J, J, F, J, E, E, I, J, J, J, I,~
## $ clarity <ord> SI2, SI1, VS1, VS2, SI2, VVS2, VVS1, SI1, VS2, VS1, SI1, VS1, ~
## $ depth   <dbl> 61.5, 59.8, 56.9, 62.4, 63.3, 62.8, 62.3, 61.9, 65.1, 59.4, 64~
## $ table   <dbl> 55, 61, 65, 58, 58, 57, 57, 55, 61, 61, 55, 56, 61, 54, 62, 58~
## $ price   <int> 326, 326, 327, 334, 335, 336, 336, 337, 337, 338, 339, 340, 34~
## $ x       <dbl> 3.95, 3.89, 4.05, 4.20, 4.34, 3.94, 3.95, 4.07, 3.87, 4.00, 4.~
## $ y       <dbl> 3.98, 3.84, 4.07, 4.23, 4.35, 3.96, 3.98, 4.11, 3.78, 4.05, 4.~
## $ z       <dbl> 2.43, 2.31, 2.31, 2.63, 2.75, 2.48, 2.47, 2.53, 2.49, 2.39, 2.~
\end{verbatim}

Another simple function that you may use regularly is the
\texttt{colnames()} function. It returns a list of column names from
your dataset. You can check out this function by running the code chunk
below:

\begin{Shaded}
\begin{Highlighting}[]
\FunctionTok{colnames}\NormalTok{(diamonds)}
\end{Highlighting}
\end{Shaded}

\begin{verbatim}
##  [1] "carat"   "cut"     "color"   "clarity" "depth"   "table"   "price"  
##  [8] "x"       "y"       "z"
\end{verbatim}

After running the code chunk, you may have noticed a number in brackets.
This number helps you count the number of columns in your dataset. If
you have data with lots of columns and \texttt{colnames()} prints the
results on multiple lines, each line will have a number in brackets at
the start of the line indicating what number column that is! So, for
example, ``carat'' is the first column in the \texttt{diamonds} dataset.
On the second line, there is the number seven in brackets; ``price'' is
the seventh column.

\hypertarget{step-3-cleaning-data}{%
\subsection{Step 3: Cleaning data}\label{step-3-cleaning-data}}

One of the most frequent tasks you will have to perform as an analyst is
to clean and organize your data. \texttt{R} makes this easy! There are
many functions you can use to help you perform important tasks easily
and quickly.

For example, you might need to rename the columns, or variables, in your
data. There is a function for that: \texttt{rename().} You can check out
how it works in the chunk below:

\begin{Shaded}
\begin{Highlighting}[]
\FunctionTok{rename}\NormalTok{(diamonds, }\AttributeTok{carat\_new =}\NormalTok{ carat)}
\end{Highlighting}
\end{Shaded}

\begin{verbatim}
## # A tibble: 53,940 x 10
##    carat_new cut       color clarity depth table price     x     y     z
##        <dbl> <ord>     <ord> <ord>   <dbl> <dbl> <int> <dbl> <dbl> <dbl>
##  1      0.23 Ideal     E     SI2      61.5    55   326  3.95  3.98  2.43
##  2      0.21 Premium   E     SI1      59.8    61   326  3.89  3.84  2.31
##  3      0.23 Good      E     VS1      56.9    65   327  4.05  4.07  2.31
##  4      0.29 Premium   I     VS2      62.4    58   334  4.2   4.23  2.63
##  5      0.31 Good      J     SI2      63.3    58   335  4.34  4.35  2.75
##  6      0.24 Very Good J     VVS2     62.8    57   336  3.94  3.96  2.48
##  7      0.24 Very Good I     VVS1     62.3    57   336  3.95  3.98  2.47
##  8      0.26 Very Good H     SI1      61.9    55   337  4.07  4.11  2.53
##  9      0.22 Fair      E     VS2      65.1    61   337  3.87  3.78  2.49
## 10      0.23 Very Good H     VS1      59.4    61   338  4     4.05  2.39
## # i 53,930 more rows
\end{verbatim}

Here, the function is being used to change the name of \texttt{carat} to
\texttt{carat\_new}. This is a pretty basic change, but
\texttt{rename()} has many options that can help you do more complex
changes across all of the variables in your data.

For example, you can rename more than one variable in the same
\texttt{rename()} code. The code below demonstrates how:

\begin{Shaded}
\begin{Highlighting}[]
\FunctionTok{rename}\NormalTok{(diamonds, }\AttributeTok{carat\_new =}\NormalTok{ carat, }\AttributeTok{cut\_new =}\NormalTok{ cut)}
\end{Highlighting}
\end{Shaded}

\begin{verbatim}
## # A tibble: 53,940 x 10
##    carat_new cut_new   color clarity depth table price     x     y     z
##        <dbl> <ord>     <ord> <ord>   <dbl> <dbl> <int> <dbl> <dbl> <dbl>
##  1      0.23 Ideal     E     SI2      61.5    55   326  3.95  3.98  2.43
##  2      0.21 Premium   E     SI1      59.8    61   326  3.89  3.84  2.31
##  3      0.23 Good      E     VS1      56.9    65   327  4.05  4.07  2.31
##  4      0.29 Premium   I     VS2      62.4    58   334  4.2   4.23  2.63
##  5      0.31 Good      J     SI2      63.3    58   335  4.34  4.35  2.75
##  6      0.24 Very Good J     VVS2     62.8    57   336  3.94  3.96  2.48
##  7      0.24 Very Good I     VVS1     62.3    57   336  3.95  3.98  2.47
##  8      0.26 Very Good H     SI1      61.9    55   337  4.07  4.11  2.53
##  9      0.22 Fair      E     VS2      65.1    61   337  3.87  3.78  2.49
## 10      0.23 Very Good H     VS1      59.4    61   338  4     4.05  2.39
## # i 53,930 more rows
\end{verbatim}

Another handy function for summarizing your data is
\texttt{summarize().} You can use it to generate a wide range of summary
statistics for your data. For example, if you wanted to know what the
mean for \texttt{carat} was in this dataset, you could run the code in
the chunk below:

\begin{Shaded}
\begin{Highlighting}[]
\FunctionTok{summarize}\NormalTok{(diamonds, }\AttributeTok{mean\_carat =} \FunctionTok{mean}\NormalTok{(carat))}
\end{Highlighting}
\end{Shaded}

\begin{verbatim}
## # A tibble: 1 x 1
##   mean_carat
##        <dbl>
## 1      0.798
\end{verbatim}

These functions are a great way to get more familiar with your data and
start making observations about it. But sometimes, previewing tables
isn't enough to understand a dataset. Luckily, \texttt{R} has
visualization tools built in.

\hypertarget{step-4-visualizing-data}{%
\subsection{Step 4: Visualizing data}\label{step-4-visualizing-data}}

With \texttt{R,} you can create data visualizations that are simple and
easy to understand or complicated and beautiful just by changing a bit
of code. \texttt{R} empowers you to present the same data in so many
different ways, which can help you create new insights or highlight
important data findings. One of the most commonly used visualization
packages is the \texttt{ggplot2} package, which is loaded automatically
when you install and load \texttt{tidyverse.} The \texttt{diamonds}
dataset that you have been using so far is a \texttt{ggplot2} dataset.

To build a visualization with \texttt{ggplot2} you layer plot elements
together with a \texttt{+} symbol. You will learn a lot more about using
\texttt{ggplot2} later in the course, but here is a preview of how easy
and flexible it is to make visuals using code:

\begin{Shaded}
\begin{Highlighting}[]
\FunctionTok{ggplot}\NormalTok{(}\AttributeTok{data =}\NormalTok{ diamonds, }\FunctionTok{aes}\NormalTok{(}\AttributeTok{x =}\NormalTok{ carat, }\AttributeTok{y =}\NormalTok{ price)) }\SpecialCharTok{+}
  \FunctionTok{geom\_point}\NormalTok{()}
\end{Highlighting}
\end{Shaded}

\includegraphics{Lesson3_Sandbox_files/figure-latex/unnamed-chunk-10-1.pdf}

The code above takes the \texttt{diamonds} data, plots the carat column
on the X-axis, the price column on the Y-axis, and represents the data
as a scatter plot using the \texttt{geom\_point()} command.

\texttt{ggplot2} makes it easy to modify or improve your visuals. For
example, if you wanted to change the color of each point so that it
represented another variable, such as the cut of the diamond, you can
change the code like this:

\begin{Shaded}
\begin{Highlighting}[]
\FunctionTok{ggplot}\NormalTok{(}\AttributeTok{data =}\NormalTok{ diamonds, }\FunctionTok{aes}\NormalTok{(}\AttributeTok{x =}\NormalTok{ carat, }\AttributeTok{y =}\NormalTok{ price, }\AttributeTok{color =}\NormalTok{ cut)) }\SpecialCharTok{+}
  \FunctionTok{geom\_point}\NormalTok{()}
\end{Highlighting}
\end{Shaded}

\includegraphics{Lesson3_Sandbox_files/figure-latex/unnamed-chunk-11-1.pdf}

Wow, that's a busy visual! Sometimes when you are trying to represent
many different aspects of your data in a visual, it can help to separate
out some of the components. For example, you could create a different
plot for each type of cut. \texttt{ggplot2} makes it easy to do this
with the \texttt{facet\_wrap()} function:

\begin{Shaded}
\begin{Highlighting}[]
\FunctionTok{ggplot}\NormalTok{(}\AttributeTok{data =}\NormalTok{ diamonds, }\FunctionTok{aes}\NormalTok{(}\AttributeTok{x =}\NormalTok{ carat, }\AttributeTok{y =}\NormalTok{ price, }\AttributeTok{color =}\NormalTok{ cut)) }\SpecialCharTok{+}
  \FunctionTok{geom\_point}\NormalTok{() }\SpecialCharTok{+}
  \FunctionTok{facet\_wrap}\NormalTok{(}\SpecialCharTok{\textasciitilde{}}\NormalTok{cut)}
\end{Highlighting}
\end{Shaded}

\includegraphics{Lesson3_Sandbox_files/figure-latex/unnamed-chunk-12-1.pdf}

You will learn many other ways of working with \texttt{ggplot2} to make
functional and beautiful visuals later on. For now, hopefully you
understand that it is both flexible and powerful!

\hypertarget{step-5-documentation}{%
\subsection{Step 5: Documentation}\label{step-5-documentation}}

You have been working in an \texttt{R\ markdown} file, which allows you
to put code and writing in the same place. Markdown is a simple language
for adding formatting to text documents. For example, all of the section
headers have been formatted by adding \texttt{\#\#} to the beginning of
the line. Markdown can be used to format the text in other ways, such as
creating bulleted lists:

\begin{itemize}
\tightlist
\item
  So if you have a list of things
\item
  Or want to write bullets for another reason
\item
  You can do that using markdown
\end{itemize}

When you have written, executed, and documented your code in an
\texttt{R\ markdown} document like this, you can use the \texttt{knit}
button in the menu bar at the top of the editing pane to export your
work to a beautiful, readable document for others.

\hypertarget{activity-wrap-up}{%
\subsection{Activity Wrap-up}\label{activity-wrap-up}}

You have had a chance to explore more \texttt{R} tools that you can
start using on your own. You learned how to install and load
\texttt{R\ packages}; functions for viewing, cleaning, and visualizing
data; and using \texttt{R\ markdown}to export your work. Feel free to
continue exploring these functions in the rmd file, or use this code in
your own RStudio project space. As you practice on your own, consider
how \texttt{R} is similar and different from the tools you have already
learned in this program, and how you might start using it for your own
data analysis projects. \texttt{R} provides a lot of flexibility and
utility that can make it a key tool in a data analyst's tool kit.

Make sure to mark this activity as complete in Coursera.

\end{document}
